%-------------------------------------
% Resume in LaTeX (XeLaTeX)
% Author:  HF Yuan
% Project: https://github.com/xyz-yuanhf/yuan-resume
% Base on: https://github.com/mattyHerzig/mattys_resume
%          http://www.jianxu.net/en/
% License: MIT
%------------------------------------

\documentclass[a4paper, 10pt]{article}
\usepackage{myresume}  % Style package

% --------------------  START  --------------------
\begin{document}

% -------------------- HEADING --------------------
\begin{flushright}
  \setstretch{0.8} % slightly tighter
  \item {\faGraduationCap\ \href{https://scholar.google.ca/citations?view_op=list_works&hl=en&user=AYYyliAAAAAJ}{\textcolor{myblue}{Google Scholar}}}
  \item {\faGithub\ \href{https://github.com/YueFei0403}{\textcolor{myblue}{github.com/YueFei0403}}}
  \item {\faEnvelope\ \textcolor{myblue}{yuefei9943@gmail.com}}
  \item {\faPhone\ \textcolor{myblue}{(+1) 647-866-7311}}
\end{flushright}\vspace{-40pt}

\begin{flushleft}
  % Name
  {\Calluna \fontsize{30pt}{30pt}\selectfont \textsc{Yue Fei}}
  \quad
  {\Calluna \fontsize{14.5pt}{14.5pt}\selectfont \textsc{she/her}} \\[4pt]

  % Tagline
  {\small \textit{Machine-Learning \& Hardware-Aware Optimization • Parallel Programming/HPC • Data Pipelines \& CI/CD}} 

  % Divider line
  \noindent\rule{\textwidth}{0.1pt}
\end{flushleft}


% -------------------- EDUCATION --------------------
\sectionBlock{
\section{Education}
}{
\justifying
\eduHeading
  {University of Toronto}
  {Master of Engineering, Electrical Engineering}
  {Sep. 2022 -- Jun. 2024}
\itemListStart
  \myItem{Emphasis: Communications}
  \myItem{Advisor: \href{https://www.comm.utoronto.ca/~rsadve/}{Prof. Raviraj Adve}}
  \myItem{MEng Thesis: “\href{https://github.com/YueFei0403/UMa_5G_AoA}{Pilot Training - Angle of Arrival and Channel Estimation in 5G
Network}”}
\itemListEnd

\eduHeading
  {University of Toronto}  
  {Bachelor of Applied Science, Electrical Engineering}
  {Sep. 2017 -- Jun. 2022}
\itemListStart
  \myItem{Capstone Project: Convolutional neural network NPU Overlay (MobileNetV1) for FPGA (Intel Stratix 10 NX 2100)}
  \myItem{Advisors: \href{https://www.eecg.utoronto.ca/~vaughn/}{Prof. Vaughn Betz} and \href{https://andrewboutros.github.io/}{Andrew Boutros}}
\itemListEnd
}

% -------------------- PUBLICATIONS --------------------
\sectionBlock{
\section{Publications\\(Peer-Reviewed Conference)}
}{
\pubListStart
\justifying
\item Arash Ahmadian, Louis S.P. Liu, \textbf{Yue Fei}, Konstantinos N. Plataniotis; Mahdi S. Hosseini. Pseudo-Inverted Bottleneck Convolution for Darts Search Space. \textit{IEEE International Conference on Acoustics, Speech and Signal Processing (ICASSP)}, 2023.
\item Abnash Bassi, \textbf{Yue Fei}, Gilead Posluns, Mark C. Jeffrey. Optimized Priority Scheduling for Faster Scalable Belief Propagation. \textit{The Association for the Advancement of Artificial Intelligence (AAAI) [In Submission]}, 2026.
\pubListEnd
}

% -------------------- AWARDS & HONORS --------------------
\sectionBlock{
\section{Awards and Honors}
}{
\awardListStart
\justifying
\item \emph{Dean's Honour List}  \hfill \textit{2017 Fall, 2018 Winter, 2018 Fall, 2021 Fall, \& 2022 Winter}
\item \emph{Edward S. Rogers Sr. Department Betz Entrance Scholarship (\$5,000)} \hfill \textit{2017}
\awardListEnd
}

% -------------------- Certificate --------------------
\sectionBlock{
\section{Certificate}
}{
\certificateListStart
\item \emph{Certificate in Engineering Business}  \hfill \textit{Jun. 2022}
\certificateListEnd
}

% -------------------- TALKS --------------------
\sectionBlock{
\section{Invited Talks}
}{
\talkItem
  {Panel: \textit{Demystifying Machine Learning}}
  {Talk: \textit{From Channels to States — Machine Learning in the Language of Communication and Control}}
  {QWomen San Diego — Qualcomm Internal Panel Discussion}
  {Mar.\ 2025}
}

% -------------------- SKILLS --------------------
\sectionBlock{
\section{Technical Skills}
}{
\skillListStart
\item \textbf{Programming:} Python (\textbf{Pandas} for SQL-style data joins), C/C++, MATLAB, Julia, Arm Assembly, Perl
\item \textbf{Data Processing \& Automation:} CI/CD (Jenkins) pipelines for large IC/IP regressions — \textbf{enabled same-day dashboards vs.\ 1–2-day manual}; Makefile-based build/test flows
\item \textbf{ML \& Optimization:} PyTorch, GRU-RNN, MLP, Attention mechanism, Q-Learning (Reinforcement Learning), Convex Optimization (fractional \& quadratic-transform), Sampling-based Source Coding
\item \textbf{Parallel Programming:} \textbf{Multithreading} (OpenMP-style loops, SIMD), custom thread mgmt, \textbf{multi-queue scheduling} for scalable workloads
\item \textbf{Tools \& Environments:} Git, Linux/Unix shells, Vim/GVim, Mur$\Phi$ Model Checker, SimpleScalar simulators

\skillListEnd
}

% -------------------- INDUSTRY EXPERIENCE --------------------
\sectionBlock{
\section{Industry Experience}
}{
\vspace{0.5pt}
\jobHeading
  {Qualcomm}{Markham, Canada}{Jun. 2024 -- Jul. 2025}
\itemListStart
  \myItem{Verified \textbf{UWB} receiver path and improved startup performance by reducing LNA charging delay \textbf{90\%} ($20$ns $\rightarrow$ $2$ns).”}
  \myItem{Validated \textbf{WLAN CP-PLL synthesizer} loop and built \textbf{UVM}-compatible test plans spanning 500+ channel indices, ensuring robust coverage across 2G, 5G, and emerging 5G alternative bands.}
  \myItem{Developed \textbf{multi-head GRU-based RNN} for receiver gain line-up optimization, where each head learns one analog block (LNA, GM, TIA, BQ, PGA). Transformed a complex combinatorial tuning problem into a scalable learning-based approach, easing designer effort.}
  \myItem{Built a physics-inspired \textbf{MLP} that predicts \textbf{VCO} capacitance from control inputs, removing the need for RF/analog designers to manually tune capacitors for $1000+$ frequency targets.}
\itemListEnd
\jobHeading
  {Alphawave Semi}{Toronto, Canada}{May 2020 -- Jun. 2021}
\itemListStart
  \myItem{Developed \textbf{UVM} testbenches to verify \textbf{SerDes} (clocking, datapath, \textbf{SRAM}), expanding functional coverage across \textbf{50+ scenarios}.}
  \myItem{Enhanced CI/CD automation to support \textbf{15× growth} in regression testing (scaling from \textbf{4 to 60+ projects}), improving efficiency and reliability as the company expanded.}
\itemListEnd
}


% ----------------- Selected Projects -----------------
\sectionBlock{
\section{Selected Projects}
}{
\vspace{0.5pt}
{\small Highlighted academic, research, and technical projects spanning ML, optimization, signal processing, HPC, and architecture.}
}

% ---------- ML & Optimization ----------
\sectionBlock{
{\textbf{ML \& Optimization}\\[-2pt]
\scriptsize Modeling and optimization for wireless networks and semantic coding}
}{
\projHeading{Convex \& Fractional Programming for Multi-Cell MIMO Beamforming}{Sep. 2023 -- Dec. 2023}
\itemListStart
  \myItem{Applied \textbf{fractional-programming} and \textbf{quadratic-transform} optimization to improve multi-cell \textbf{MIMO beamforming}, boosting convergence and power-constrained sum-rate performance.}
\itemListEnd

\projHeading{Sampling-Based Semantic Source Coding (One-Shot Info Theory)}{Jan. 2023 -- May. 2023}
\itemListStart
  \myItem{Implemented \textbf{Poisson functional representation}, \textbf{rejection sampling}, \textbf{importance sampling}, and hybrid Poisson\,+\,dithered-quantization for 6G semantic source-coding in MATLAB.}
\itemListEnd

\projHeading{Transformer \& Embedding Visualization (Research Assistant)}{Jul. 2021 -- Sep. 2021}
\itemListStart
  \myItem{Explored \textbf{RNN-based Transformer models} and \textbf{Attention mechanisms} for NLP tasks.}
  \myItem{Applied \textbf{PCA-based embedding visualization}—as used in \textbf{GloVe}—to project high-dimensional embeddings into \textbf{2D/3D space} using \textbf{Python (NumPy, Matplotlib)}, enabling intuitive inspection of semantic clusters.}
\itemListEnd
}

% ---------- Signal Processing ----------
\sectionBlock{
{\textbf{Signal Processing}\\[-2pt]
\scriptsize Estimation and detection for modern wireless systems}
}{
\projHeading{Angle of Arrival (AoA) \& Channel Estimation}{Jan. 2023 -- May. 2023}
\itemListStart
  \myItem{Implemented \textbf{MUSIC}, \textbf{DFT}, and \textbf{Matrix-Pencil} eigenvalue methods in MATLAB for AoA estimation under large-scale and Rayleigh-fading channels; \textbf{Matrix-Pencil delivered ≈2 dB gain in low-SNR regimes with 16-antenna arrays}, outperforming MUSIC and DFT for highly-correlated signals.}
\itemListEnd

\projHeading{LTE Signal Processing}{Jan. 2024 -- Apr. 2024}
\itemListStart
  \myItem{Processed captured LTE signals for \textbf{time/frequency sync}, \textbf{OFDM demodulation}, and \textbf{pilot-power analysis} in MATLAB; resampled 40 MHz front-end data to 30.72 MHz LTE rate and validated \textbf{PSS/SSS} detection.}
\itemListEnd
}

% ---------- Parallel / HPC ----------
\sectionBlock{
{\textbf{Parallel \& HPC}\\[-2pt]
\scriptsize Multi-threaded acceleration and memory-coherence design}
}{
\projHeading{Parallel Beamforming \& Cache-Coherence Foundations for Scalable Compute}{Jan. 2024 -- May. 2024}
\itemListStart
  \myItem{Accelerated medical-imaging by \textbf{7×} (17 s → 2.5 s) via \textbf{16-thread data-parallel} ultrasound beamforming with SIMD intrinsics and memory optimizations (\texttt{restrict}, single-write); validated correctness (RMS error < 1e-16) and scalability (1–16 threads) — a paradigm relevant to \textbf{data-parallel LLM training}.}
  \myItem{Designed and verified a \textbf{3-hop directory cache-coherence protocol} (MSI/MESI) in Murϕ; optimized with \textbf{Exclusive (E)} state to eliminate bus transactions (\textbf{0 MB} overhead vs 80 MB baseline), formalized \textbf{7 invariants} (e.g., single-writer ownership), and handled FSM edge-case scenarios for large-scale shared-memory systems.}
\itemListEnd
}

% ---------- Coding Theory ----------
\sectionBlock{
{\textbf{Coding Theory}\\[-2pt]
\scriptsize Error-correcting codes for reliable communications}
}{
\projHeading{Graph-Based Error-Correcting Codes}{Sep. 2023 -- Dec. 2023}
\itemListStart
  \myItem{Implemented \textbf{LDPC}, fountain/LT, and \textbf{Polar encoders/decoders} over binary-erasure channel; developed custom simulators in Julia and MATLAB.}
\itemListEnd

\projHeading{Convolutional Codes \& Viterbi Decoder}{Jan. 2023 -- May. 2023}
\itemListStart
  \myItem{Built a rate-½ \textbf{convolutional encoder} and \textbf{Viterbi decoder} in MATLAB with custom trellis structures; validated against built-in tools and demonstrated 2-bit-error correction on a noisy BSC channel.}
\itemListEnd
}

% ---------- Architecture ----------
\sectionBlock{
{\textbf{Computer Architecture}\\[-2pt]
\scriptsize Speculation mechanisms relevant to LLM decoding}
}{
\projHeading{Computer Architecture Coursework}{Sep. 2021 -- Dec. 2021}
\itemListStart
  \myItem{Developed a \textbf{5-stage pipelined CPU} with \textbf{hazard detection \& forwarding}, implemented a \textbf{perceptron-based branch predictor}, \textbf{Tomasulo out-of-order execution}, \textbf{Bouquet prefetcher}, and \textbf{MSI-directory cache-coherence protocol}; these experiences deepened my understanding of \textbf{pipeline parallelism for distributed LLM training} and how \textbf{speculative execution} in CPUs parallels \textbf{LLM speculative decoding} for faster inference.}
\itemListEnd
}

\end{document}

